\documentclass[11pt]{article}
\usepackage[english]{babel}
\linespread{1.0}
\usepackage{slantsc}
\usepackage[hmargin=1in,vmargin=3cm]{geometry}
%\input{./bild_mk2.sty}
\input{./default.sty}
\usepackage[largesc]{newtxtext}
\usepackage[scaled=0.85]{DejaVuSansMono}
\usepackage[vvarbb, nosymbolsc]{newtxmath}
\usepackage{multicol}
\usepackage[cal=esstix, scr=rsfso]{mathalfa}
\usepackage{bm}
\usepackage{verbatim}
%\numberwithin{equation}{section}
\renewcommand{\arraystretch}{1.0}
%\usepackage{titlecaps}
%\usepackage[explicit]{titlesec}
%\usepackage{titletoc}
%\titleformat{\section}{\Large\sffamily\bfseries}{\thesection. }{1.5ex}{#1}
%\titleformat{\subsection}{\large\sffamily\bfseries}{\thesubsection. }{1.0ex}{#1}
%\titleformat{\subsubsection}{\sffamily\bfseries}{\thesubsubsection\ }{0.5ex}{#1}

%\titlespacing*{\section}{0em}{1.0em}{0.5em}
%\titlespacing*{\subsection}{0em}{1.0em}{0.5em}
%\titlespacing*{\subsubsection}{0pt}{0.5em}{0.5em}
%
%\pagestyle{fancyplain}
%\renewcommand{\headrulewidth}{0pt}
%\fancyhf{} \fancyfoot[R]{\thepage}
%\renewcommand{\title}[3]{\begin{flushleft}\LARGE\sffamily\bfseries{#1}\\[1ex]
%    \normalfont\sffamily \large #2 \hfill \normalsize #3 \end{flushleft}}
%
%\usepackage{hyperref}
%\hypersetup{colorlinks=true,linkcolor=Black,citecolor=Black,urlcolor=MplBlue}
%\newcommand{\ds}{\displaystyle}
%\renewcommand{\vec}[1]{\harpoonacc{#1}}
%\renewcommand{\t}{\dagger}
%\newcommand{\cip}[2]{\left(#1\middle|#2\right)\:}
%\newcommand{\cmel}[3]{\left(#1 \middle| #2 \middle| #3 \right)\:}
%\newcommand{\Coul}{\mathcal{J}}
%\newcommand{\Exch}{\mathcal{K}}
%\newcommand{\coverb}[2]{\overbrace{#1}^{\mathclap{#2}}}
%\newcommand{\cunderb}[2]{\underbrace{#1}_{\mathclap{#2}}}
\begin{document}
\title{The SCF Workshop Notes 2}{Lucas Aebersold}{\today}
\setcounter{section}{2}
\subsection{How to Gaussian Integrate}
%pg 153 -> gbasis 
Let's take a quick step back and ask the very same question that dawned on me when learning about this stuff ``What is the actual mathematical form of the thing we're solving? I mean TF do I actually put into a code to compute this stuff''. Let's first refamiliarize ourselves with Gaussian basis sets, and then derive our expressions using Gaussian orbitals. 

In molecular calculations, you generally use a fixed molecular coordinate system, so that the basis functions are centered at position vectors $\bb{R}_A$. The value at a position vector $\bb{r}$, of a function centered at $\bb{R}_A$ will depend on $\bb{r} - \bb{R}_A$, thus we write a general basis function as $\phi_\mu (\bb{r} - \bb{R}_A)$ to denote that it is centered at $\bb{R}_A$
\begin{figure}[H]
	\centering
\begin{tikzpicture}[
hole/.style={
	circle,
	fill=Black,
	inner sep=0pt,
	minimum width=1.2mm
},
	empty/.style={
	inner sep=0pt,
	outer sep=0pt,
},
]

\draw[semithick] (0,0,0) to (0, 0, 3); 
\draw[semithick] (0,0,0) to (0, 3, 0); 
\draw[semithick] (0,0,0) to (3, 0, 0); 

\node[empty] (O) {}; 
\node[hole] (rRA) at ++(60:4cm) {};
\coordinate (Ac) at ++(30:3.0cm); 
\node[circle, ball color=MplBlue, opacity=0.3, minimum size=1.52cm] (E) at (Ac) {}; 

\node[hole,label=-30:$A$] (A) at ++(30:3.0cm) {};

%\draw[name path=AB] (A) to node[near end, hole, label=180:$P$] (P) {} (B); 
%\draw[name path=CD] (C) to node[near start, hole, label=0:$Q$] (Q) {} (D); 

\draw[->] (O) to node[auto, swap] {$\mathbf{R}_A$} (A);
\draw[->] (O) to node[auto] {$\mathbf{r}$} (rRA);
\draw[->] (A) to node[auto, swap] {$\mathbf{r} - \mathbf{R}_A$} (rRA);

\end{tikzpicture}
\caption{Coordinate system representation for the atom centered Gaussian}
\end{figure}
Our typical unnormalized Gaussian friend looks like
\begin{equation}\label{key}
\tilde{g}_{1s} (\bb{r} - \bb{R}_A) = e^{-\alpha | \bb{r} - \bb{R}_A|^2} 
\end{equation}
The normalized 1$s$ \textit{Gaussian-type function}, centered at $\bb{R}_A$, can be written as 
\begin{equation}\label{key}
\phi_{1s}^{\mathrm{GF}} (\alpha, \bb{r} - \bb{R}_A) = \qty(\frac{2\alpha}{\pi})^{3/4}e^{-\alpha|\bb{r} - \bb{R}_A|^2} 
\end{equation}
The integrals take the generic form 
\begin{equation}\label{}
\cip{\mu_A \nu_B}{\lambda_C \sigma_D}= \int \phi_\mu^{A*}(\bb{r}_1) \phi_\nu^{B}(\bb{r}_1) r_{12}^{-1} \phi_\lambda^{C*} (\bb{r}_2) \phi_\sigma^D (\bb{r}_2) \dd{\bb{r}_1 d\bb{r}_2} 
\end{equation}
which will involve taking the product of Gaussian functions, fortunately this product is just another Gaussian centered at the midpoint between the two other centers
\begin{equation}
\phi_{1s}^{\mathrm{GS}}(\alpha, \mathbf{r} - \mathbf{R}_A) \phi_{1s}^{\mathrm{GS}}(\beta, \mathbf{r} - \mathbf{R_B}) = K_{AB} \phi_{1s}^{\mathrm{GS}}(p, \mathbf{r}- \mathbf{R}_p)
\end{equation}
where this constant $K_{AB}$ is 
\begin{equation}\label{}
K_{AB} = \qty(\frac{2\alpha\beta}{(\alpha + \beta) \pi})^{3/4} \exp(-\frac{\alpha\beta}{\alpha + \beta}) \exp(|\mathbf{R}_A - \mathbf{R}_B|^2)
\end{equation}
The new exponent of the Gaussian centered at $\mathbf{R}_P$ is 
\begin{equation}\label{}
p = \alpha + \beta 
\end{equation}
and the third center $P$ is on a line joining the centers $A$ and $B$
\begin{equation}\label{key}
\mathbf{R}_P = (\alpha \mathbf{R}_A + \beta \mathbf{R}_B)/(\alpha + \beta) 
\end{equation}
\begin{figure}[H]
	\centering
	\includegraphics[width=0.5\textwidth]{gauss_overlaps}
	\caption{Product of two Gaussian} 
\end{figure}
\begin{equation}\label{}
\cip{\mu_A \nu_B}{\lambda_C \sigma_D}= K_{AB} K_{CD} \int \phi_{1s}^{\mathrm{GS}}(p, \mathbf{r}_1 - \bb{R}_p)r_{12}^{-1} \phi_{1s}^{\mathrm{GS}}(q, \bb{r}_2 - \bb{R}_Q) \dd{\bb{r}_1 d\bb{r}_2} 
\end{equation}
The formula for a series of Gaussian, known as contracted GFs is
\begin{equation}\label{key}
\phi_\mu^{\mathrm{CGF}}(\bb{r} - \bb{R}_A) = \sum_{p = 1}^L d_{p\mu} \phi_p^{\mathrm{GF}}(\alpha_{p\mu}, \bb{r} - \bb{R}_A) 
\end{equation}
where $d_{p\mu}$ are the contraction coefficients
\subsubsection{Integral Forms of 1s Gaussians}
The $\bb{S}$ overlap matrix has the basic form that when integrated gives us the following equation
\begin{align*}
S = \cip{A}{B} &= \int \tilde{g}_{1s}(\bb{r}_1 - \bb{R}_A)\tilde{g}_{1s}(\bb{r}_1 - \bb{R}_B) \dd{\bb{r}_1} \\
&= \qty(\frac{\pi}{(\alpha + \beta)})^{3/2} \exp(-\frac{\alpha\beta}{(\alpha + \beta)}|\bb{R}_A - \bb{R}_B|^2)
\end{align*}
Kinetic energy integral will give us this expression 
\begin{align*}
\cmel{A}{-\frac{1}{2}\laplacian }{B} &=\int \tilde{g}_{1s}(\bb{r}_1 - \bb{R}_A) (-\frac{1}{2}\laplacian) \tilde{g}_{1s}(\bb{r}_1 - \bb{R}_B) \dd{\bb{r}_1} \\
&= \frac{\alpha\beta}{(\alpha + \beta)} \left[ 3 - \frac{2\alpha\beta}{(\alpha + \beta)} | \bb{R}_A - \bb{R}_B|^2 \right]\left[\frac{\pi}{(\alpha + \beta)}\right]^{3/2} \exp(-\frac{\alpha\beta}{(\alpha + \beta)} |\bb{R}_A - \bb{R}_B|^2)
\end{align*}
For the nuclear attraction integral, I'm going to have to pull a Cukier and refer you to pg. 412--414 of S\&O for the exact derivation. 

During the derivation (which involves Fourier transforms!---you'll never escape them, submit to their study or perish), we end up needing to introduce the $F_0$ function, defined as
\begin{equation}\label{key}
F_0 (t) = t^{-1/2} \int_0^{t^{1/2}} e^{-y^2}\dd{y} 
\end{equation}
It is related to the error function by 
\begin{equation}\label{key}
F_0 (t) = \frac{1}{2} (\pi/t)^{1/2} \erf(t^{1/2}) 
\end{equation}
which itself is related to the Gamma function and so on, anyway, our nuclear attraction integral then becomes
\begin{equation}\label{key}
\cmel{A}{-\frac{Z_C}{r_{1C}^{
}}}{B} = -\frac{2\pi}{(\alpha + \beta)} Z_C \exp(-\frac{\alpha \beta}{(\alpha + \beta)}|\bb{R}_A - \bb{R}_B|^2) F_0\qty((\alpha + \beta)|\bb{R}_P - \bb{R}_C|^2)
\end{equation}
Now consider the horrid two-electron repulsion integral 
\begin{align*}
\cip{AB}{CD} &= \exp(-\frac{\alpha\beta}{(\alpha + \beta)} |\bb{R}_A - \bb{R}_B|^2 - \frac{\gamma \delta}{(\gamma + \delta)} |\bb{R}_C - \bb{R}_D|^2 ) \\
&\qquad \times \int \exp(-p|\bb{r}_1 - \bb{R}_P|^2) r_{12}^{-1} \exp(-q |\bb{r}_2 - \bb{R}_Q|^2) \dd{\bb{r}_1 d\bb{r}_2} \\
 &= \frac{2\pi^{5/2}}{(\alpha + \beta)(\gamma + \delta)(\alpha + \beta + \gamma + \delta)^{1/2}}\\
 &\qquad \times \exp(-\frac{\alpha\beta}{(\alpha + \beta)} |\bb{R}_A - \bb{R}_B|^2 - \frac{\gamma \delta}{(\gamma + \delta)} |\bb{R}_C - \bb{R}_D|^2) F_0 \qty(\frac{(\alpha + \beta)(\gamma + \delta)}{(\alpha + \beta + \gamma + \delta)} |\bb{R}_P - \bb{R}_Q|^2)
\end{align*}

\begin{figure}[H]
	\centering
	\begin{tikzpicture}[
	hole/.style={
		circle,
		fill=Black,
		inner sep=0pt,
		minimum width=1.2mm
	},
	]
	\node[hole,label=-60:$B$] (B) {}; 
	\node[hole,label=30:$A$] (A) at ++(80:3cm) {};
	
	\node[hole,label=0:$D$] (D) at ($(B) + (-10:4cm)$) {}; 
	\node[hole,label=30:$C$] (C) at ($(D) + (100:3.5cm)$) {};
	\draw[name path=AB] (A) to node[near end, hole, label=180:$P$] (P) {} (B); 
	\draw[name path=CD] (C) to node[near start, hole, label=0:$Q$] (Q) {} (D); 
	\draw[->] (P) to node[below=0.5cm, midway, inner sep=0pt, outer sep=0pt] {$\mathbf{R}_Q - \mathbf{R}_P$} (Q); 
	\end{tikzpicture}
	\caption{The six centers involved in the two-electron repulsion integral. }
\end{figure}

\end{document}
